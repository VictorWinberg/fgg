\documentclass{fgg}

\title{Protokoll Helgmys}
\author{Fina Grabbarnas Grupp}

\date{\formatdate{14}{11}{2020}}
\datum{Lördagen den 14 november 2020 kl. 18.00}
\plats{Zacharias HQ, Ellstorp, Malmö}

\begin{document}
\makehf
\maketitle

\begin{narvaro}
	\person{Zacharias Berg}{Hovmästare}
	\person{Ellie Anonym}{Sexmästare}
	\person{Anton Björkman}{Kommunikatör}
	\person{Victor Winberg}{Ytterst oklart}
\end{narvaro}

\begin{protokoll}
	\paragraf{1}{Tid och Sätt}
	Även då viss förvirring inför Helgmys fanns, där diskussion rörande \emph{Lund} tidigare existerat, blev det till slut \textbf{fastställt} att \emph{Malmö} blev helgmysets definita plats.
	\paragraf{2}{Öppnande}
	\emph{Anton} och \emph{Winberg} våldgästade Zacharias och Ellie, där viss problematik uppstod för tillgång in till lägenheten. Men efter en distraktion i form utav en promenad, blev \textbf{tillgång beviljad}.
	\paragraf{3}{Middag}
	Bjöds på av värdparet, som var \textbf{mycket uppskattad} av alla parter.
	\paragraf{4}{Spel och Snack}
	Efter ett spel \textbf{Absolut Överens}, med intensiva och jämna runder, som avslutades i en \textbf{draw} blev det ny röstning av nästa spel. Spelet som valdes var \textbf{Citadels}. Om man sett \emph{Star Wars: Episode III – Revenge of the Sith} så kan jag förmedla att denna spelomgång i Citadels kan liknas vid \textbf{Anakin's Betrayal}, mer behöver inte tilläggas. De berörda parterna känner till historien.
	\paragraf{5}{Avslut}
	Vi skiljer oss åt för att återförenas snart igen.
\end{protokoll}

\signature{Väl mött}{Victor Winberg}{Obi-Wan Kenobi}
\signature{}{Anton Björkman}{R2-D2}
\signature{}{Zacharias Berg}{Anakin}

\end{document}